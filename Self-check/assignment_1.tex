\documentclass[a4paper,10pt]{article}

\usepackage{amsthm,amsfonts,amsmath,amssymb,amscd}
\usepackage{indentfirst}
\usepackage[usenames]{color}
\usepackage{color}
\usepackage{colortbl}

\usepackage[singlelinecheck=off,center]{caption}
\usepackage{soul}

\usepackage{cite}

\usepackage[plainpages=false,pdfpagelabels=false]{hyperref}
\definecolor{linkcolor}{rgb}{0.9,0,0}
\definecolor{citecolor}{rgb}{0,0.6,0}
\definecolor{urlcolor}{rgb}{0,0,1}
\hypersetup{
    colorlinks, linkcolor={linkcolor},
    citecolor={citecolor}, urlcolor={urlcolor}
}

\usepackage{graphicx}
\graphicspath{{images/}}

\makeatother
%%%%%%%%%%%%%%%%%%%%%%%%%%%%%%%%%%%%%%%%%%%%%%%%%%%%%%%%%%%%%%%%%%%%%%%%%%%%%%%%%%%

\newcommand{\ket}[1]{|#1\rangle}
\newcommand{\bra}[1]{\langle#1|}
\newcommand{\ketbra}[2]{\ket{#1}\langle#2|}
\newcommand{\tr}{\ensuremath{{\rm Tr}}}

\begin{document}

\begin{center}
\Large{{\bf Self-check problems on\\  Quantum information processing}\\{\it Part 1: Basics of Quantum Mechanics}}\\
\vspace{5pt}
\large{Last update: \today}
\end{center}

\subsection*{Problem 1}
Let $\mathcal{H}$ be a $d$-dimensional Hilbert space.
Denote it computational basis states as
\begin{equation}
	\ket{0} = \begin{bmatrix}
	1 \\ 0 \\ \vdots \\ 0
	\end{bmatrix}, \quad 
	\ket{1} = \begin{bmatrix}
	0 \\ 1 \\ \vdots \\ 0
	\end{bmatrix}, \quad
	\ldots, \quad
	\ket{d-1} = \begin{bmatrix}
	0 \\ 0 \\ \vdots \\ 1
	\end{bmatrix}.
\end{equation}
Let
\begin{equation}
	\ket{\psi} = \begin{bmatrix}
		C_0 \\ C_1 \\ \vdots \\C_{d-1}
	\end{bmatrix}, \quad
	M = \begin{bmatrix}
		m_{0,0} & m_{0,1} & \ldots & m_{0,d-1} \\
		m_{1,0} & m_{1,1} & \ldots & m_{1,d-1} \\		
		\vdots & \vdots & \ddots & \vdots \\		
		m_{d-1,0} & m_{d-1,1} & \ldots & m_{d-1,d-1} \\		
	\end{bmatrix}.
\end{equation}
be some matrix.

Check the following facts.
\begin{enumerate}
	\item $\bra{i}j\rangle = \delta_{i,j}$, where $\delta_{i,j}$ is Kronecker symbol.
	\item $\sum_{i=0}^{d-1}\ket{i}\bra{i}={\bf 1}$, where ${\bf 1}$ is the identity matrix.
	\item $C_i = \bra{i} \psi \rangle$.
	\item $\ket{\psi}=\sum_{i=0}^{d-1} C_i \ket{i}$.
	\item $M_{i,j} = \bra{i}M\ket{j}$.
	\item $M=\sum_{i,j}M_{i,j}\ket{i}\bra{j}$.
	\item $M\ket{\psi}=\sum_{k,l=0}^{d-1}M_{k,l}C_l\ket{k}$.
\end{enumerate}


\subsection*{Problem 2}
Consider standard Pauli matrices
\begin{equation}
	\sigma_x = \begin{bmatrix}
	0 & 1 \\ 1 & 0 \\
	\end{bmatrix}, \quad
		\sigma_y = \begin{bmatrix}
	0 & -{\rm i} \\ {\rm i} & 0 \\
	\end{bmatrix}, 
	\sigma_z = \begin{bmatrix}
		1 & 0 \\ 0 & -1 \\
	\end{bmatrix}, \quad
	{\bf 1} = \begin{bmatrix}
		1 & 0 \\ 0 & 1
	\end{bmatrix}.
\end{equation}
Find their eigenvalues and eigenvectors. 
Check that each matrix can be represented in the form $\sum_{i=0}^{1}\lambda_i\ket{\psi_i}\bra{\psi_i}$, where $\{\lambda_i\}$ and $\{\ket{\psi_i}\}$ are eigenvalues and eigenvectors correspondingly.
Pay special attention to non-uniqueness of ${\bf 1}$ spectral decomposition.
Show the eigenvectors on the Bloch sphere.


\subsection*{Problem 3}
Let $\{\ket{\psi_i}\}_{i=1}^d$ be a set of orthonormal states ($\bra{\psi_i}\psi_j\rangle=\delta_{i,j}$).
Check that
\begin{equation}
	U = \begin{bmatrix}
		\ket{\psi_1} & \ket{\psi_2} & \ldots & \ket{\psi_d}
	\end{bmatrix}
\end{equation}
is a unitary matrix.
Check that a spectral decomposition $M=\sum_{i=1}^d\lambda_i\ket{\psi_i}\bra{\psi_i}$ can be written in the form
\begin{equation}
	M = UDU^{\dagger},
\end{equation}
where $D$ is diagonal matrix with $D_{i,i}=\lambda_i$


\subsection*{Problem 4}
Let $f(\xi): \mathbb{R}\rightarrow \mathbb{R}$ be some function.
Here we would like to extend $f$ on Hermitian matrices.
Consider a Hermitian matrix $M$ with a spectral decomposition $M=\sum_k \lambda_k \ket{\psi_k}\bra{\psi_k}$.
Let
\begin{equation}
	f(\xi) = \sum_{i=0}^{\infty}C_i\xi^i 
\end{equation}
be a Taylor decomposition of $f$.
Check that
\begin{equation}
	f(M) := \sum_{i=0}^{\infty}C_iM^i= \sum_k f(\lambda_k) \ket{\psi_k}\bra{\psi_k}.
\end{equation}


\subsection*{Problem 5}
Check that $\ket{\psi(t)}=U(t)\ket{\psi_0}=\exp({-{\rm i}Ht})\ket{\psi_0}$ is solution of the Schroedinger equation
\begin{equation}
	{\rm i}\frac{d}{dt}\ket{\psi(t)}=H\ket{\psi(t)}
\end{equation}
with initial condition $\ket{\psi(0)}=\ket{\psi_0}$.


\subsection*{Problem 6}
Consider a Hamiltonian
\begin{equation}
	H = \frac{\hbar \omega}{2} \sigma_x
\end{equation}
and an initial state
\begin{equation}
	\ket{\psi_0} = \begin{bmatrix}
		1 \\ 0
	\end{bmatrix}.
\end{equation}
\begin{enumerate}
	\item Find an evolution operator $U(t)$.
	\item Find an evolution of initial state in the Schroedinger picture $\ket{\psi(t)}$. Show it on the Bloch sphere.
	\item Find an evolution of mean value of $\sigma_z$.
	\item Find an evolution of $\sigma_z$ in the Heisenberg picture $\sigma_z^H(t)$. Express it in terms of other Pauli matrices. 
	\item Check that both pictures provide the same mean value of $\sigma_z$.
\end{enumerate}


\subsection*{Problem 7}
Check that making a Hamiltonian transformation
\begin{equation}
	H \rightarrow H + \alpha {\bf I},
\end{equation}
where $\alpha$ is real and ${\bf I}$ is identity matrix of corresponding dimension, results in acquiring additional phase for the evolution operator.
Show that this phase does not affect the result of evolution of a density matrix.


\subsection*{Problem 8}
Let $\{\ket{n}_{A}\}_{n=0}^{d_{A}-1}$ and $\{\ket{m}_{B}\}_{m=0}^{d_{B}-1}$ be computational bases of finite-dimensional Hilbert spaces $\mathcal{H}_{A}$ and $\mathcal{H}_{B}$ ($d_{A}$ and $d_{B}$ are corresponding dimensions).
Write an expression for position of unit element inside the vector $\ket{i}_{A}\otimes \ket{j}_{B}$ in terms of $i$ and $j$.
Check applicability your result for two-qubit case considered in the lectures.


\subsection*{Problem 9}
Consider a two qubit state with density matrix
\begin{equation}
	\rho_{AB} = \begin{bmatrix}
		\rho_{00} & \rho_{01} & \rho_{02} & \rho_{03} \\
		\rho_{10} & \rho_{11} & \rho_{12} & \rho_{13} \\
		\rho_{20} & \rho_{21} & \rho_{22} & \rho_{23} \\
		\rho_{30} & \rho_{31} & \rho_{32} & \rho_{33} \\
	\end{bmatrix} =
	\begin{bmatrix}
		X & Y \\
		Z & W
	\end{bmatrix},
\end{equation}
where
\begin{equation}
	X = 	\begin{bmatrix}
		\rho_{00} & \rho_{01} \\
		\rho_{10} & \rho_{11}
	\end{bmatrix}, \quad
	Y = 	\begin{bmatrix}
		\rho_{02} & \rho_{03} \\
		\rho_{12} & \rho_{13}
	\end{bmatrix}, \quad
	Z = 	\begin{bmatrix}
		\rho_{20} & \rho_{21} \\
		\rho_{30} & \rho_{31}
	\end{bmatrix}, \quad
	W = 	\begin{bmatrix}
		\rho_{22} & \rho_{23} \\
		\rho_{32} & \rho_{33}
	\end{bmatrix}.
\end{equation}
Show that
\begin{equation}
	\rho_{A} = \tr_{B}\rho_{AB} = 
	\begin{bmatrix}
		\tr X & \tr Y \\
		\tr Z & \tr W
	\end{bmatrix}, \quad
	\rho_{B} = \tr_{A}\rho_{AB}  = X+W.
\end{equation}


\subsection*{Problem 10}
Consider three-qubit W-state\footnote{Note that this kind of states appears in Rydberg blockade.}
\begin{equation}
	\ket{{\rm W}}_{ABC} = \frac{1}{\sqrt{3}}(\ket{100}_{ABC} + \ket{010}_{ABC} + \ket{001}_{ABC}).
\end{equation}
Write its Schmidt decomposition with respect to bipartite partitioning $A:BC$ (consider qubits $B$ and $C$ as single object).
What is a Schmidt rank of the resulting decomposition? 
Show that $A$ is not maximally entangled with $BC$.
Design a three-qubit state $\ket{\Psi}_{ABC}$ where the first qubit $A$ is maximally entangled with a pair $BC$.


\subsection*{Problem 11}
Consider the following two-qubit states:
\begin{equation} \label{eq:states-for-purification}
	\rho_1= \frac{{\bf 1}}{2}\otimes \frac{{\bf 1}}{2}, \quad \rho_{2}=\frac{1}{2}(\ket{00}\bra{00}+\ket{11}\bra{11}), \quad \rho_{3}=\ket{\Phi^{+}}\bra{\Phi^{+}},
\end{equation}
where
\begin{equation} \label{eq:phiplus}
	\ket{\Phi^{+}}=\frac{1}{\sqrt{2}}(\ket{00}+\ket{11}).
\end{equation}
Check that reduced state of both qubits coincide for all three states.
Find mean values of observables
\begin{equation}
	\sigma_{z}^{A}\equiv \sigma_{z}\otimes {\bf 1}, \quad \sigma_{z}^{B}\equiv {\bf 1}\otimes  \sigma_{z}, \quad \sigma_{z}\otimes\sigma_{z}, \quad \sigma_{x}\otimes\sigma_{x},
\end{equation}
for each state $\rho_{i}$.


\subsection*{Problem 12}
Find purifications of all three states~\eqref{eq:states-for-purification}.
Do the state of ancillary purifying system (``environment'') correlate with the purified state?


\subsection*{Problem 13}
Show that mixture of mixed separable states is also mixed separable state.


\subsection*{Problem 14}
Consider maximally entangled state $\ket{\Phi^{+}}$ from Eq.~\eqref{eq:phiplus}.
Write $\rho=\ket{\Phi^{+}}\bra{\Phi^{+}}$ in the form
\begin{equation}
	\rho = \sum_{i} p_{i} \rho_{A}^{(i)} \otimes \rho_{B}^{(i)}, \quad p_{i}\geq 0, \quad \sum_{i}p_{i}=1,
\end{equation}
where $\rho_{A}^{(i)}$ and $\rho_{B}^{(i)}$ are \emph{arbitrary} $2\times2$ matrices.
Remember definition of separable mixed states. 
Draw conclusions :-)


\end{document}
